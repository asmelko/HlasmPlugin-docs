\chapter{Project execution}

In the following chapter is represented execution of the High Level Assembler Plugin software project. 
We analyze the problem difficulty, break it into tasks and estimate time requirements of particular tasks.
We further describe the team and work organization.

\section{Tasks}
We analyzed the problem and split it into several tasks. At the time of writing this document implementation is already in the 24. week of our schedule and we have working prototype. Therefore the presented tasks are specified. 

Tasks were assigned to individual team members during stand ups. The tasks and their assignment (team member name initials in the parentheses following task name) is presented in the Gantt diagram(s)  (\ref{fig:gantt1}, \ref{fig:gantt2}, \ref{fig:gantt3}). 
Project implementation is planned for nine months. 

\section{Collaboration}
The team consists of five members. Collaboration within the team is essential for successful completion of the project. We use variety of means to achieve this. 

Our team works with agile software development. To aid this we use visual process management system Kanban. The team meets every week together with our supervisor at stand ups. Team discusses current status of particular tasks with their owners, review progress and plan work for next week.

For communication between team members is used online tool Slack.

\newgeometry{a4paper,left=1in,right=1in,top=1in,bottom=1in,nohead}

\begin{landscape}
	\begin{figure}
		\centering
		\begin{ganttchart}[vgrid={draw=none, dotted}, x unit = 1.2cm]{1}{12}
		
		\gantttitlelist{"1. month", "2. month", "3. month"}{4} \\
		\gantttitlelist{1,...,12}{1} \\
		
		\ganttbar{HLASM language analysis (A \& Ma)}{1}{8} \\
		\ganttbar{Parser libraries research (P)}{1}{4}\\
		\ganttbar{IDEs research (L \& Mi)}{1}{4}\\
		
		\ganttbar{LSP POC (Mi)}{5}{8} \\
		\ganttbar{VSCode client POC(Mi)}{5}{8}\\
		\ganttbar{Lexer (L \& P)}{5}{10}\\
		\ganttbar{Parser (A \& Ma)}{5}{12}\\
		
		\ganttbar{Client semantic highlighting (M)}{9}{12}\\
		\ganttbar{Assembler checker (L)}{9}{12}\\
		\ganttbar{Conditional assembly instructions (A)}{9}{12}\\
		\ganttbar{Conditional assembly expressions (P)}{9}{12}\\
		\ganttbar{Debugger POC (M)}{9}{12}
		
		
		\end{ganttchart}
    \caption{Tasks for months 1 -- 3}
	\label{fig:gantt1}
	\end{figure}
\end{landscape}


\begin{landscape}
	\begin{figure}
		\centering
		\begin{ganttchart}[vgrid={draw=none, dotted}, x unit = 1.2cm]{1}{12}
			
			\gantttitlelist{"4. month", "5. month", "6. month"}{4} \\
			\gantttitlelist{13,...,24}{1} \\
			\ganttbar{Conditional assembly LSP features (Ma)}{1}{4}\\
			\ganttbar{Machine instruction checker (L)}{1}{12}\\
			\ganttbar{Macro expansion (A)}{1}{4}\\
			
			\ganttbar{Copy instruction (A)}{5}{8}\\
			\ganttbar{Machine expressions (Mi)}{5}{8}\\
			\ganttbar{Client-server continuation handling (Ma)}{5}{8}\\
			
			\ganttbar{DC instruction (Mi \& P)  $\rightarrow$}{9}{12}\\
			\ganttbar{Ordinary symbols (A \& P) $\rightarrow$}{9}{12}\\
			\ganttbar{Diagnostics (L)}{9}{12}
			
			\ganttvrule[vrule offset=0.8]{today}{12}
			
		\end{ganttchart}
		\caption{Tasks for months 4 -- 6}
		\label{fig:gantt2}
	\end{figure}
\end{landscape}

\begin{landscape}
	\begin{figure}
		\centering
		\begin{ganttchart}[vgrid={draw=none, dotted}, x unit = 1.2cm]{1}{12}
			
			\gantttitlelist{"7. month", "8. month", "9. month"}{4} \\
			\gantttitlelist{25,...,36}{1} \\
			
			
			\ganttbar{$\rightarrow$ DC instruction (Mi \& P)}{1}{4}\\
			\ganttbar{$\rightarrow$ Ordinary symbols (A \& P)}{1}{4}\\
			\ganttbar{Ordinary LSP features (Ma)}{1}{4}\\
			\ganttbar{Code coverage (L)}{1}{8}\\
			
			\ganttbar{Benchmarking (Ma)}{5}{8}\\
			\ganttbar{Testing (all)}{5}{12}\\
			
			\ganttbar{Documentation (all)}{9}{12}\\
		
			
			\ganttvrule[vrule offset=0.2]{today}{1}
			
		\end{ganttchart}
		\caption{Tasks for months 7 -- 9}
		\label{fig:gantt3}
	\end{figure}
\end{landscape}

\restoregeometry

mirko:

milestony

gantt

prirazeni lidi k projektum

udelejte si cas na psani dokumentace

je fajn mit contingency plan, co delat kdyz se to dojebe nebo ltery ficury jsou jak prioritni

\todo{je fajn vsechno tohle podeprit tim ze mate prototyp, a jak se na nej bude navazovat. Rozhodne do specifikace uz nemuzete psat ze budete volit parser a ide, protoze to tady ma bejt specifikovany.}


