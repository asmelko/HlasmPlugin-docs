\chapter{Requirements}

This chapters outlines the main features that the project should implement to be a working product. All the features below have been discussed with professional HLASM developers and have been agreed on by the company.

The final product is a VS Code extension, downloadable from the Market Place. The extension contains all executables/libraries that are needed for the project to work correctly on the most popular platforms. No other prerequisites should be needed.

Modularity of the project is another important requirement. 

The parsing library provides 2 kinds of API: complex one that mirrors the LSP specification and a simple one that accepts a text along with a dependency-resolver object and returns diagnostics. 

The language server implements the LSP standard, hence it can be easily reintegrated within other IDEs such as Eclipse Che.

\section{Language features}
This part provides a brief overview of the parts of HLASM that are included in the working product.

\begin{itemize}
\item The project recognizes the syntax of HLASM and parses it into predefined structures. 

\item It is able to interpret high-level parts of the assembler. There are conditional assembly instructions for the code generation and the macro expansion:
\begin{itemize}
    \item AIF
    \item AGO
    \item MACRO, MEND, MEXIT
    \item ACTR
    \item SETA, SETB, SETC
    \item ANOP
    \item LCLA, LCLB, LCLC
    \item GBLA, GBLB, GBLC
    \item AEJECT
    \item ASPACE
\end{itemize}
and assembler instructions for code layout determination:
\begin{itemize}
    \item EQU 
    \item DC 
    \item DS 
    \item DSECT, CSECT, RSECT 
    \item LOCTR
\end{itemize}

\item All instructions (including machine instructions) are being semantically and syntactically checked for correct operand format usage. However, the run-time register and memory values are not being analyzed as there is no machine instruction interpretation.

\item Due to the possibility of using external files, a dependency search must be implemented. To do so, two configuration files are necessary to provide locations of depending files. They look as shown in \cref{lst:pgm} and \cref{lst:grps}. 

The \texttt{proc\_grps} file contains names of \texttt{processor groups} with the relative paths to the external files. The \texttt{pgm\_conf} matches the processor groups to the source code files. This configuration procedure is similar to the one used on real mainframes. We provide a built-in pre-configuration option for these files.

\item Another aspect of HLASM which has to be considered and handled are fixed-sized lines. This makes the parsing more difficult as the continuation character can be placed in different cases differently. We also add a continuation handling option to the IDE which mostly consists of non-movable continuation characters, i.e. if the user types something in front of the continuation character, it stays on place.

\item On top of that, \texttt{macro tracer} is implemented using \texttt{Debugging Adapter Protocol}. Following standard debugging procedure, the user can step through the code generation while watching the conditional assembly variables and the call stack. This tool is extremely useful as tracing the macro expansions manually gets tedious quickly.
\end{itemize}

\pagebreak
\begin{listing}
\begin{verbatim}
{
  "pgroups": [
    {
      "name": "P1",
      "libs": [
        "copies",
        "libs"
      ]
    }
  ]
}
\end{verbatim}
\caption{proc\_grps.json}
\label{lst:grps}
\end{listing}

\begin{listing}
\begin{verbatim}
{
  "pgms": [
    {
      "program": "source_code",
      "pgroup": "P1"
    }
  ]
}
\end{verbatim}
\caption{pgm\_conf.json}
\label{lst:pgm}
\end{listing}

\section{LSP features}
This section demonstrates the possible uses of the extension on the client side. LSP provides a list of well-defined features. The project implements the following:

\begin{itemize}
	\item Go to definition for all symbols, macro definitions and copy members
	\item Find all references for all symbols, macro definitions and copy members
	\item Completion for instructions, defined symbols and macros
	\item Hover for symbol attributes, their locations, contents and other useful information depending on the symbol type
	\item Diagnostics for syntax and semantic errors and warnings
	\item (Extra to LSP) Server-side Highlighting for all symbols  
\end{itemize}

The highlighting is not a standard part of the LSP, nonetheless it is a needed addition. Due to the complexity of HLASM, a typical syntax highlighting is not sufficient.
