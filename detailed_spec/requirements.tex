\chapter{Requirements}

This chapters outlines the main features that the project should implement to be a working product. All the features below have been discussed with professional HLASM developers and have been agreed on by the company.

The final product is a VS Code extension, downloadable from the Market Place. The extension contains all executables/libraries that are needed for the project to work correctly on the most popular platforms. No other prerequisites should be needed.

Modularity of the project is another important requirement. 

The parsing library provides 2 kinds of API: complex one that mimicks the LSP specification and a simple one that accepts text and dependency-resolver object and returns diagnostics. 

The language server implements the LSP standard, hence it can be easily reintegrated within other IDEs such as Eclipse Che.

\section{Language features}
This part provides a brief overview of the parts of HLASM that will be implemented in  the project.

First of all, the project recognizes the syntax of HLASM (including continuations) and parses it into predefined structures. 

It is able to interpret high-level parts of the assembler such as the conditional assembly instructions for the code generation, assembler instruction for the code layout determination and the macro expansion.

For the dependency search, a configuration file is necessary to provide locations of depending files. This configuration procedure is similar to the one used on real mainframes.

On top of that, all instructions (including machine instructions) are being semantically and syntactically checked for correct operand format usage. However, the operand contents are not being checked as there is no machine instruction interpretation.



\section{LSP features}
This section demonstrates the possible uses of the extension on the client side. LSP provides a list of well-defined features. The project implements the following:

\begin{itemize}
	\item Go to definition for all symbols, macro definitions and copy members
	\item Find all references for all symbols, macro definitions and copy members
	\item Completion for instructions, defined symbols and macros
	\item Hover for symbol attributes, their locations, contents and other useful information depending on the symbol type
	\item Diagnostics for syntax and semantic errors and warnings
	\item (Extra to LSP) Server-side Highlighting for all symbols  
\end{itemize}

The highlighting is not a standard part of the LSP, nonetheless it is a needed addition. Due to the complexity of HLASM, a typical syntax highlighting is not sufficient.
