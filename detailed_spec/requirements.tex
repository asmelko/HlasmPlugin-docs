\chapter{Project scope}

This chapter reviews the specific goals of our project.
All the features below have been discussed with professional HLASM developers and have been agreed on by the company.

This project aims to produce a VS Code extension, downloadable from the Market Place. The extension contains all executables/libraries that are needed for the project to work correctly on the most popular platforms. No other prerequisites should be needed.

Modularity of the software is another important requirement. 

The parsing library provides 2 kinds of API: a complex one that mirrors the Language Server Protocol\footnote{\url{https://microsoft.github.io/language-server-protocol/}} (LSP) specification and a simple one that accepts a text along with a dependency-resolver object and returns diagnostics. 

The language server implements the LSP standard, hence it can be easily reintegrated within other IDEs such as Eclipse Che.

\section{Supported language features}
This part provides a brief overview of the parts of HLASM that should be included in the final software.

\begin{description}
\item [Syntax Parser] The parser recognizes the syntax of HLASM and processes it into predefined structures. 

\item [High-level interpretation] The library interprets high-level parts of the assembler. Following are the conditional assembly instructions for code generation and macro expansion:
\begin{itemize}
    \item AIF
    \item AGO
    \item MACRO, MEND, MEXIT
    \item ACTR
    \item SETA, SETB, SETC
    \item ANOP
    \item LCLA, LCLB, LCLC
    \item GBLA, GBLB, GBLC
    \item AEJECT
    \item ASPACE
\end{itemize}
and assembler instructions for code layout determination:
\begin{itemize}
    \item EQU 
    \item DC 
    \item DS 
    \item DSECT, CSECT, RSECT 
    \item LOCTR
\end{itemize}

\item [Code Validation] The back-end semantically and syntactically checks all instructions (including machine instructions) for correct operand format usage. However, it does not analyze the run-time register and memory values, as there is no machine instruction interpretation.

\item [Dependency Search] Usage of external files in HLASM is a highly common phenomenon. On mainframes, HLASM programs are built according to its JCL\footnote{Job Control Language, instructs the system how to run a specific task} file, which contains a list of libraries. When the compiler encounters an undefined instruction or a \texttt{COPY} instruction, it does a top-down search through the whole list. Both ways of invoking a dependency search are demonstrated in \cref{lst:search}. As a large portion of the programs use the same libraries, defining these JCL files gets repetitive. Therefore, the build and source management system Endevor creates an abstraction above the libraries and groups them into \texttt{processor groups}. As a result, JCL offers an option to identify the libraries to be included by their processor group's ID instead of listing them all manually.

We adapt this system to our needs and define 2 configuration files. The first one mirrors the behavior of Endevor and defines the processor groups. The second one matches the source codes to these processor groups.

\item [Continuation Handling] Fixed-size lines are another aspect of HLASM that needs to be handled. They make the parsing more difficult, as the position of the continuation character may vary. 

We also add a continuation handling option to the IDE, which mostly consists of non-movable continuation characters, i.e. if the user types in front of the continuation character, it stays in place.

\item [Macro Tracer] To trace the code generation, the user can step through the code while watching the contents of variables and the call stack using \texttt{Macro tracer}. We implement \texttt{Debug Adapter Protocol} so that the process ressembles standard debugging. This tool is extremely useful as tracing the macro expansions manually gets tedious quickly.
\end{description}

\pagebreak
\begin{listing}
\begin{verbatim}
        MAC1       1,1                   
        COPY       COPY1
\end{verbatim} 
\caption{An example of both ways the HLASM program may invoke dependency search.}
\label{lst:search}
\end{listing}

\section{Supported LSP features}
This section demonstrates the possible uses of the extension on the client side. LSP provides a list of well-defined features. The project implements the following:

\begin{itemize}
	\item \texttt{Go to definition} command for all symbols, macro definitions and copy members\footnote{Copy, along with macro expansion, is a mean to include another external file, invoked by \texttt{COPY} instruction. Comparing to macro, copy does not neither need to start nor end with any specific instruction and the invoking \texttt{COPY} instruction is simply replaced by the COPY file's contents.}
	\item \texttt{Find all references} command for all symbols, macro definitions and copy members
	\item Completion for instructions, defined symbols and macros
	\item Hover for symbol attributes, their locations, contents and other useful information depending on the symbol type
	\item Diagnostics for syntax and semantic errors and warnings
	\item (Custom LSP extension) Server-side Highlighting for all symbols  
\end{itemize}

The highlighting is not a standard part of the LSP, nonetheless it is a needed addition. Due to the complexity of HLASM, a typical syntax highlighting is not sufficient. Consider following examples:

\lstdefinelanguage{HLASM}{
    keywords = [1]{REMARK},
    keywords = [2]{SAM31,LR,AGO,AIF,ICTL},
    keywords = [3]{J,SYMBOL},
     keywords = [4]{IGNORED},
     keywords = [5]{X},
    keywordstyle = [1]\color{blue},
    keywordstyle = [2]\color{red},
    keywordstyle = [3]\color{white},
    keywordstyle = [4]\color{magenta},
    keywordstyle = [5]\color{cyan}
}
\lstset{
numbers=left, 
numberstyle=\small, 
numbersep=8pt, 
frame = single, 
language=HLASM, 
framexleftmargin=15pt,
backgroundcolor = \color{lightgray}
}
\begin{itemize}
    \item The language server recognizes operand formats of different instructions. The most simple example is the \texttt{SAM31} instruction, which does not have any operands, while the \texttt{LR} instruction takes two. So the identifier right after \texttt{SAM31} is colored blue as a remark. 
    
\begin{lstlisting}
    SAM31 REMARK
    LR    1,1   REMARK
\end{lstlisting}

\item The code skipped by the conditional assembly is not colored and stays white.
\begin{lstlisting}
    AGO .HERE
    J   SYMBOL
.HERE AIF
\end{lstlisting}

\end{itemize}

